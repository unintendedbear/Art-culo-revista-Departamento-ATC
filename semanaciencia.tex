\documentclass[runningheads,a4paper]{llncs}

\usepackage[utf8]{inputenc}
\setcounter{tocdepth}{3}
\usepackage{graphicx}
\usepackage{multirow}
\usepackage{rotating}
\usepackage{url}

\newcommand{\keywords}[1]{\par\addvspace\baselineskip
\noindent\keywordname\enspace\ignorespaces#1}

\providecommand{\tabularnewline}{\\}

\begin{document}

\mainmatter

\title{¿Por qu\'{e} no estamos todas las que somos? Cr\'{o}nica de la semana de la ciencia.}

\author{Paloma de las Cuevas \and M. Isabel Garc\'{i}a-Arenas\inst{1}}

\institute{Dept. of Computer Architecture and Technology, University
of Granada, Spain}

\maketitle

% Abstract

\begin{abstract}

It is known that the percentage of women in STEM degrees has fallen to minimums, being less than 20\% in the best case scenario. This work summarises the experiences and opinions obtained during an activity called ``Spanish pun I do not know how to translate yet'', held during the ``Science Week'' at the ETSIIT. A group of 10 students and a teacher from upper secondary school where asked to share opinions and debate about some facts and thoughts from a gender point of view.

\keywords{keyword1, keyword2}
\end{abstract}

\section{Introduction}
\label{sec:intro}

...

\section{Background}
\label{sec:background}

...

\section{Methodology}
\label{methodology}

...

\section{Results}
\label{sec:results}

...

\section{Conclusion}
\label{conclusion}

...

\section*{Acknowledgments}

...

\bibliographystyle{splncs}
\bibliography{semanaciencia}

\end{document}
